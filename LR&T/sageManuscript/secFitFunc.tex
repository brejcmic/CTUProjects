\section{Fitness Function}
The fitness function is the essential part of the GA that assesses the solutions. The target solution's fitness function value is the minimum or maximum of the fitness function, depending on the fitness function design. There are no general rules for its range of values and only simple recommendations for its creation. The target value of average maintained illuminance and of uniformity are watched by the algorithm. The output design is considered most effective if target values are reached with minimum count of luminaires. Therefore the minimization of the luminaires count is also expected.

The first attempt to create a suitable fitness function was based on the idea, that the minimum luminaire number would be obtained just by the requirement of a target illuminance. The algorithm was supposed to reach the target value of illuminance with the highest possible uniformity. However only the restriction of getting the exact value of the illuminance was not sufficient to get the minimum count of luminaires. The demand to maximize uniformity favored solutions with slightly more luminaires than needed to fulfill the minimum illuminance requirement. Therefore the minimalization of luminaire count had to be directly included into the fitness function.

The fitness function bellow is currently used in the algorithm. Solutions that do not reach basic requirements of target illuminance or uniformity are evaluated by length of the chromosome $L$ (Equation~\ref{eq:chromLength}):

\begin{equation}
\label{eq:fitV2EUA}
	f\left(C,\overline{E}_{m}, U_0\right)= L
\end{equation}

\noindent Otherwise the fitness is evaluated by:

\begin{equation}
\label{eq:fitV2EUB}
\begin{split}
f\left(C, \overline{E}_{m}, U_0\right)&=C +\\
& + \left( 1 - \alpha\right)\cdot\frac{\overline{E}_{mT}}{\overline{E}_{m}+\epsilon} + \\
& + \alpha\cdot\frac{U_{0T}}{U_0 + \epsilon}
\end{split}
\end{equation}

\noindent where:
\begin{description}
	\item[$C$] is the count of luminaires,
	\item[$\overline{E}_{m}$] is the calculated maintained average value of illuminance,
	\item[$\overline{E}_{mT}$] is the target maintained average value of illuminance,
	\item[$U_0$] is the calculated lighting uniformity,
	\item[$U_{0T}$] is the target lighting uniformity,
	\item[$\alpha$] is a number between 0 and 1,
	\item[$\epsilon$] is a very small number, that prevents the division by 0.
\end{description}

$\alpha$ represents the designer's intention to prefer one of the parameter to another. The target ratio between relative values of the parameters can be obtained from partial derivation of the fitness function:

\begin{equation}
\label{eq:fitV2targetRatio}
R_T =\sqrt{\frac{1-\alpha}{\alpha}}
\end{equation}

\noindent There is no preference if $\alpha$ is equal to $0.5$. Only illuminance is evaluated by the fitness function for $\alpha = 0$ and only uniformity is evaluated for $\alpha = 1$. The actual value of ratio $R$ is affected by the observed parameters. Consider the case where the average illuminance is preferred ($R_T > 1$), but its value could not be changed enough by different luminaire placement. Then the uniformity is optimized and the ratio:

\begin{equation}
\label{eq:fitV2ratio}
R =\frac{\overline{E}_{m}}{\overline{E}_{mT}}\cdot\frac{U_{0T}}{U_0}
\end{equation}

\noindent might even reach a value less than~$1$. 

This fitness function is supposed to be minimized, i.e. the smaller the output of the fitness function, the higher the quality of the solution. Equation \ref{eq:fitV2EUB} can be, if close to the target values of illuminance and uniformity, separated into an integer part, given only by the count of luminaires $h_1\left(C\right)= C$ and a fraction part that is given by:

\begin{equation}
\label{eq:fitV2frac}
	h_2\left(\overline{E}_{m}, U_0\right)= \left( 1 - \alpha\right)\cdot\frac{\overline{E}_{mT}}{\overline{E}_{m}+\epsilon} + \alpha\cdot\frac{U_{0T}}{U_0 + \epsilon}
\end{equation}

The main impact of the integer part $h_1\left(C\right)$ occurs at the beginning of the optimization. Only solutions that fulfill the standard requirements can have a smaller value of fitness than the length of its chromosome. For other solutions the most effective way to minimize the outcome of Equation~\ref{eq:fitV2EUB} is by decreasing the count of luminaires. Changes of part $h_2\left(\overline{E}_{m}, U_0\right)$ are less significant, because for high counts of luminaires, the denominator of Equation~\ref{eq:fitV2frac} is much higher than the numerator. Therefore the resulting fraction is very small.

After getting the optimal count of luminaires, the $h_2\left(\overline{E}_{m}, U_0\right)$ function's output value will be close to 1 but smaller. An effective way to minimize Equation~\ref{eq:fitV2EUB} is to get the highest possible illuminance and uniformity for the given number of luminaires.

The defined fitness function was used for several runs of the GA and has always worked well. However there is a known problem in the definition of Equation~\ref{eq:fitV2EUA}. Consider the case, that none of the initial solutions would fulfill the standard requirements. Then all of them would have the same fitness value equal to chromosome length $L$. It is very probable, that after selection, crossovers and mutations at least a single solution will appear fulfilling the standard requirements. But there can also be rare cases in which the algorithm never optimizes the count of luminaires because the fitness function in these cases is independent on all parameters of their solutions. The presence of the problem is more probable for very small population sizes, a small count of generations, a small probability of mutations or for an inappropriately designed selection method.

Some parameter dependency can be added to Equation~\ref{eq:fitV2EUA}. It was however quite useless for the further described settings of the GA and the chosen selection method. It would be extremely rare if any of the solutions fulfilling the standard requirements would not appear after a couple of generations.
