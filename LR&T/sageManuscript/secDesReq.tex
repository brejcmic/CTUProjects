\section{Design Requirements and GA Settings}
The design results must meet all requirements given by standards. Therefore the first task is always to obtain at least the minimal level of illuminance and uniformity. It is obvious that several possible designs overpass the minimal values and fit the standard. Better specifications and other restrictions for the results must be specified. Some specifications arise from the designer's preferences, for example solutions with higher efficiency could be preferred. Other designer's preferences can involve:

\begin{enumerate}
	\item[(i)] minimal count of luminaires (efficiency),
	\item[(ii)] symmetric luminaire placement,
	\item[(iii)] placement restricted only to a specific area,
	\item[(iv)] placement into specifically shaped groups of luminaires,
	\item[(v)] getting the highest illuminance for the given uniformity etc.
\end{enumerate}

The result must also respect the luminaire dimensions. Placement of luminaires close to each other might be impracticable in reality.

The genetic algorithm can handle the requirements in two ways. The first incorporates the requirements into the fitness function. It is the case, where the requirements are some part of the phenotype. This is very common and it can be used for example in case of achieving target illuminance and uniformity requirements. Other specifications fitting this method are (i) and (v) from the list above. The used apporach will be described in the next section. 

It is very difficult to implement some requirements into the fitness function. Therefore the second possible way to implement requirements is through restrictions of allowed solutions. Such method fits the specifications (ii), (iii) and (iv) from the list above. The final algorithm processed only solutions with positions restricted to a defined grid. The grid was defined to respect the luminaire dimensions, therefore the problem of proximity was solved. On the other hand the placement optimum was also restricted by the grid, which is not desirable. The authors are currently solving this unpleasant feature of the used algorithm. However the grid allowed to define symmetry of the placement in a simple way.

\subsection{Symmetry}
The final algorithm solves two types of symmetry. The first type is a 2-fold rotational symmetry with respect to the center of the ceiling. Luminaires are placed only in one half of the model room's ceiling. The other half is filled by placing equivalent luminaires with respect to the center. All luminaires form pairs with identical distances from the center. It is obvious that this symmetry does not allow to get solutions with odd numbers of luminaires. This fact had to be respected also during the design of the grid of allowed positions. The algorithm was not able to handle an odd number of intersections.

The second symmetry type is a mirror symmetry with two axes. Luminaires are placed only in a quarter of the model room. The remaining luminaires are placed by mirroring. Similar to the previous case the number of luminaires will be a multiple of 4.

\subsection{Genotype}
The resulting genotype defines a logic vector of luminaire presence in specific grid intersections. The length of the chromosome depends on the number of grid intersections and on the type of symmetry:
\begin{equation}
\label{eq:chromLength}
L = \frac{N_G}{2\cdot sym}
\end{equation}
where:
\begin{description}
	\item[$N_G$] is the number of grid intersections ($m\times n$ for Table~\ref{tab:strucgenotype}),
	\item[$sym$] is the chosen symmetry that is equal to 1 for center symmetry and equal to 2 for axis symmetry.
\end{description}

\begin{table}[b]
	\renewcommand{\arraystretch}{1.3}
	\caption{Chromosome structure}
 	\label{tab:strucgenotype}
	\centering
  \begin{tabular}{| c | c | c |}
    \hline
    $\lbrace0,1\rbrace$ at $[x_1,y_1]$ & $\ldots$ & $\lbrace0,1\rbrace$ at $[x_m,y_1]$ \\
    \hline
    $\vdots$ & $\ddots$ & $\vdots$ \\
    \hline
    $\lbrace0,1\rbrace$ at $[x_1,y_n]$ & $\ldots$ & $\lbrace0,1\rbrace$ at $[x_m,y_n]$ \\
    \hline
  \end{tabular}
\end{table}

The proposed structure of the genotype lets the algorithm determine the needed number of luminaires. Therefore only target values of illuminance and uniformity have to be set and the luminous intensity distribution curve of the chosen luminaires has to be specified.

\subsection{Mutation}
Mutation ensures variability of chromosomes even after similarity between solutions has been reached. The population's chromosome similarity is a common state of the GA, because all offspring solutions are forced to converge to the best result. In the first attempts of GA realizations we used only one type of mutation. It was based on a random inversion of the luminaire presence in a chromosome. The GA worked quite well and all results met always the standard requirements. However the returned solutions had distinctly different qualities for several runs.

The mentioned mutation overcame local extremes of the fitness function in later generations only with difficulties. The later results typically identified the adequate count of luminaires, that were needed to satisfy the target illuminance. The difference in phenotypes among results over several runs was especially in a little bit different placement of luminaires. The change of a luminaire's position is associated with the inversion of two genes of opposite values of a single chromosome, i.e. of a single solution. One inversion just adds or removes a luminaire. Therefore at least two mutations of the same chromosome had to be applied to get slightly different solutions in later generations. This hardly happened when low probability of mutation was set. On the other hand high probability of mutation increased the variation of the results and the GA slew down its convergence.

To improve the GA's behavior random chromosome permutation has been added. The mutation has two stages in the final GA. At first inversion of random genes is made. Then up to three random gene pairs of a single chromosome are switched. It has been expected that the permutation would help the algorithm especially in later generations. However the permutation strongly affects the convergence being essential for solving the presented task. After adding the permutation, the results had very similar qualities. They were also superior in comparison to results obtained using simple mutation in most of the runs. The speed of convergence has been increased and finding the solution took less number of generations.
