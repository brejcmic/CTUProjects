\section{Genetic Algorithm Introduction}

Genetic algorithms (GA) belong to the group of evolutionary algorithms. Similar to living organisms solutions are represented by their genotype, representing the parameter coding and by their phenotype, representing behaviour, response and features of the solutions. The genotype is typically coded as a vector of parameters termed chromosome, with elements being described as genes~\cite{Fogel2006}. Each solution is evaluated by its phenotype and selected if the phenotype fits the constraints and conditions of an examined environment or task. The ability to survive and reproduce in a specific environment is called fitness. The selection must maintain or increase the fitness of solution sets that are called populations. Each set of solutions corresponds to a so called generation.

Every generation starts by selecting solutions with high fitness to form a set of parent solutions for the next population. After that the process of crossovers and mutations creates the next population. The crossover combines two chromosomes of parent solutions together. The one-point crossover was used in the presented solution. Parent chromosomes are randomly split at the same point and the parts following this point are switched. The mutation changes the chromosome directly. It can change the value of a selected gene for example. The paper deals with the implemented mutation in the next section. Both the crossovers and mutations are applied only with a certain probability. The probability of crossovers is typically high (more than 90~\%) in comparison to mutation, which must be low (under 10~\%). High probability of mutation makes the optimization difficult because of the high rate of change in chromosomes.
