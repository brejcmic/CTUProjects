\section{Genetic Algorithm Introduction}
The genetic algorithms (GA) are the part of the evolutionary computing. Similar to the living organism are the solutions represented by their genotype, that represents the parameter coding and by phenotype, that represents behaviour, response and features of the solutions. The genotype is typically coded as a vector of parameters termed a chromosome, with elements being described as genes~\cite{Fogel2006}. Each solution is considered according to its phenotype and selected if the phenotype fits the constraints and conditions of an examined environment or task. The ability to survive and reproduce in specific environment is called fitness. The Selection must maintain or increase the fitness of solution sets that are called populations. Each set of solutions corresponds to one so called generation.

Each generation starts with selecting solutions with high fitness to form set of parent solutions for the next population. After that the process of crossovers and mutations creates new population. The crossover combines two chromosomes of parent solutions together. The one-point crossover was used in presented solutions. Both chromosomes are randomly split at the same point and the parts that follow that point are exchanged. The mutation changes the value of the selected gene. Both the crossovers and mutations are made only with certain probability. The probability of crossovers is typically high (more than 90~\%) in comparison to mutation, which must be low (under 10~\%). High probability of mutation makes the optimization difficult because of the high rate of change in chromosomes.