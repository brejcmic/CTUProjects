% sage_latex_guidelines.tex V1.01, 11 June 2015

\documentclass[Afour,sageh,times,doublespace]{sagej}

\usepackage{moreverb,url}

\usepackage[colorlinks,bookmarksopen,bookmarksnumbered,citecolor=red,urlcolor=red]{hyperref}

\usepackage{epstopdf}

\newcommand\BibTeX{{\rmfamily B\kern-.05em \textsc{i\kern-.025em b}\kern-.08em
T\kern-.1667em\lower.7ex\hbox{E}\kern-.125emX}}

\def\volumeyear{2016}

\begin{document}

\runninghead{Bayer and Brejcha}

\title{Indoor Ligthing System Design Considering Reflections}

\author{Rudolf Bayer\affilnum{1} and Michal Brejcha\affilnum{2}}

\affiliation{\affilnum{1}CTU in Prague, Faculty of Electrical Engineering, Department of Electrical Power Engineering\\
\affilnum{2}CTU in Prague, Faculty of Electrical Engineering, Department of Electrotechnology}

\corrauth{Rudolf Bayer, CTU in Prague, Faculty of Electrical Engineering, Department of Electrical Power Engineering, Technick� 2, Praha 6, 166 27, Czech Republic}

\email{bayerrud@fel.cvut.cz}

\begin{abstract}
The paper deals with problems concerning indoor luminaire placement by genetic algorithm. In contrast to outdoor illuminance calculations multiple reflections from walls must be taken into account. Therefore a basic reflection calculation has been proposed and a genetic algorithm script tested in software MATLAB on a model room. It appeared that requirements laid out by the Czech national standards do not restrict solutions of luminaire placement too much, hence several solutions met the requirements. The most suitable solution is always chosen by the designer's preferences. Involving these preferences into the algorithmic solution is quite a big deal, nonetheless some methods how it can be accomplished are presented in the paper.
\end{abstract}

\keywords{Genetic Algorithm, Lighting, Design, Illuminance}

\maketitle

\section{Introduction}
Designing interior lighting systems for indoor working places from a photometric point of view requires fulfilling two contradictory criteria, i.e. providing enough light for persons occupying the given room at a reasonable power consumption. These and more parameters have been taken into account while composing standards such as \cite{12464}, being mandatory on the territory of the Czech Republic.

\section{Model Room Illuminance Calculation}
\section{Genetic Algorithm Introduction}
The genetic algorithms (GA) are the part of the evolutionary computing. Similary to the living organism are the solutions represented by their genotype, that represents the genetic coding and by phenotype, that represents beahiviour, response and features of the solutions. Each soution is considered according to its phenotype.
\section{Design Requirements}
\section{Fitness Function}
\section{Luminaire Placement Problems}

\end{document}
