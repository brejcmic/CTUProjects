% sage_latex_guidelines.tex V1.01, 11 June 2015

\documentclass[Afour,sageh,times,doublespace]{sagej}

\usepackage{moreverb,url}

\usepackage[colorlinks,bookmarksopen,bookmarksnumbered,citecolor=red,urlcolor=red]{hyperref}

\usepackage{epstopdf}

\newcommand\BibTeX{{\rmfamily B\kern-.05em \textsc{i\kern-.025em b}\kern-.08em
T\kern-.1667em\lower.7ex\hbox{E}\kern-.125emX}}

\def\volumeyear{2016}

\begin{document}

\runninghead{Bayer and Brejcha}

\title{Indoor Ligthing System Design Considering Reflections}

\author{Rudolf Bayer\affilnum{1} and Michal Brejcha\affilnum{2}}

\affiliation{\affilnum{1}CTU in Prague, Faculty of Electrical Engineering, Department of Electrical Power Engineering\\
\affilnum{2}CTU in Prague, Faculty of Electrical Engineering, Department of Electrotechnology}

\corrauth{Rudolf Bayer, CTU in Prague, Faculty of Electrical Engineering, Department of Electrical Power Engineering, Technick� 2, Praha 6, 166 27, Czech Republic}

\email{bayerrud@fel.cvut.cz}

\begin{abstract}
The paper deals with problems about genetic algorithm solution of indoor luminaire placement. The multiple reflections from walls must be taken into account in contrast to outdoor illuminance calculation. Therefore the basic calculation of the reflection was proposed and the genetic algorithm was as a script in software MATLAB tested on a model room. It appeared that the requirements described in standards do not restrict the solutions of the luminaire placement to much. Therefore the standards requirements are met by several solutions. The best one is always choosen by designers preferences. It is quite a big deal to invole the preferences in the algorithmic solution, but some methods, how it can be done, are also presented in the paper.
\end{abstract}

\keywords{Genetic Algorithm, Lighting, Design, Illuminance}

\maketitle

\section{Introduction}
Designing interior lighting systems for indoor working places from a photometric point of view requires fulfilling two contradictory criteria, i.e. providing enough light for persons occupying the given room at a reasonable power consumption. These and more parameters have been taken into account while composing standards such as \cite{12464}, being mandatory on the territory of the Czech Republic.

\section{Model Room Illuminance Calculation}
\section{Genetic Algorithm Introduction}
The genetic algorithms (GA) are the part of the evolutionary computing. Similary to the living organism are the solutions represented by their genotype, that represents the genetic coding and by phenotype, that represents beahiviour, response and features of the solutions. Each soution is considered according to its phenotype.
\section{Design Requirements}
\section{Fitness Function}
\section{Luminaire Placement Problems}

\end{document}
