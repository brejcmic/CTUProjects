% sage_latex_guidelines.tex V1.01, 11 June 2015

\documentclass[Afour,sagev,times,doublespace]{sagej}

\usepackage{moreverb,url}

\usepackage[colorlinks,bookmarksopen,bookmarksnumbered,citecolor=red,urlcolor=red]{hyperref}

\usepackage{epstopdf}

\newcommand\BibTeX{{\rmfamily B\kern-.05em \textsc{i\kern-.025em b}\kern-.08em
T\kern-.1667em\lower.7ex\hbox{E}\kern-.125emX}}

\def\volumeyear{2016}

\begin{document}

\runninghead{Bayer and Brejcha}

\title{Indoor Ligthing System Design Considering Reflections}

\author{Rudolf Bayer\affilnum{1} and Michal Brejcha\affilnum{2}}

\affiliation{\affilnum{1}CTU in Prague, Faculty of Electrical Engineering, Department of Electrical Power Engineering\\
\affilnum{2}CTU in Prague, Faculty of Electrical Engineering, Department of Electrotechnology}

\corrauth{Rudolf Bayer, CTU in Prague, Faculty of Electrical Engineering, Department of Electrical Power Engineering, Technick� 2, Praha 6, 166 27, Czech Republic}

\email{bayerrud@fel.cvut.cz}

\begin{abstract}
The paper deals with problems concerning indoor luminaire placement by genetic algorithm. In contrast to outdoor illuminance calculations multiple reflections from walls must be taken into account. Therefore a basic reflection calculation has been proposed and a genetic algorithm script tested in software MATLAB on a model room. It appeared that requirements laid out by the Czech national standards do not restrict solutions of luminaire placement too much, hence several solutions met the requirements. The most suitable solution is always chosen by the designer's preferences. Implementing these preferences into the algorithmic solution is quite a big deal, nonetheless some methods how it can be accomplished are presented in the paper.
\end{abstract}

\keywords{Genetic Algorithm, Lighting, Luminaire placement, Illuminance}

\maketitle

\section{Introduction} \label{sec:intro}
%symetrie rozmístění svítidel, požadavky 12464, zvolená místnost 5*10 metrů, 500 lx bez oken, odraznosti započítány
While designing lighting systems for indoor working places the designer must take into account several requirements, of which the main would be to provide enough light for the given purpose of the interior space at a reasonable power consumption. These and more requirements are set by the Czech national standards \cite{12464} mandatory on the territory of the Czech republic.

Within the framework of this project a test model room has been chosen of dimensions 5~m $\times$ 10~m, 4~meters high with luminaires 4~metres above the floor. The model room's chosen purpose has been to provide for handwriting, writing on typewriters, reading and processing data according to reference number 5.26.2~\cite{12464}. There are several conditions required by the standard that have to be met by the lighting system:

%\begin{description}
%	\item[$\overline{E}_{m}$] Maintained Average Illuminance of 500 lux
%	\item[$UGR_{L}$] Unified Glare Rating 19
%	\item[$U_{0}$] Lighting Uniformity 0.6
%	\item[$R_{a}$] General Color Rendering Index 80
%\end{description}

\begin{itemize}
	\item $\overline{E}_{m}$ Maintained Average Illuminance of 500 lux
	\item $UGR_{L}$ Unified Glare Rating 19
	\item $U_{0}$ Lighting Uniformity 0.6
	\item $R_{a}$ General Color Rendering Index 80
\end{itemize}

The model room will meet the stated requirements if the reference plane's average illuminance will not drop below $\overline{E}_{m}$ over the course of operation of the lighting system. Calculating the initially needed illuminance values can be achieved by using the Maintenance Factor $MF$~\cite{CIE97}. $MF$ defines the depreciation of the reference plane's illuminance at the end of the maintenance period. For the model room $MF$ has been chosen to be 0.75. The reference plane~\cite{12464} represents for the chosen model room's function writing desks and has therefore been placed horizontally 75~cm above the entire floor. According to \cite{360011} the measurement grid should start 1~m away from walls with spacing of 0.5~m to 2~m. For this project we chose a more detailed grid covering the whole reference plane. To calculate $UGR_{L}$ the task area and occupants' view directions must be known, which is beyond the scope of this project. Furthermore a room of this size using ordinary office luminaires would most probably not yield values higher than the limiting top $UGR_{L}$ value. $R_{a}$ is a parameter of light sources and luminaires dependent on their light spectrum, is not dependent on the test room and will therefore not be included into calculations.
\section{Photometric Value Calculation}
%Výpočet mnohonásobných odrazů, algoritmus - odrazy

\section{Model Room Illuminance Calculation}
\section{Genetic Algorithm Introduction}
The genetic algorithms (GA) are the part of the evolutionary computing. Similary to the living organism are the solutions represented by their genotype, that represents the genetic coding and by phenotype, that represents beahiviour, response and features of the solutions. Each soution is considered according to its phenotype.
\section{Design Requirements}
\section{Fitness Function}
\section{Luminaire Placement Problems}

\begin{thebibliography}{99}

\bibitem{12464}
\v{C}SN EN 12464-1. \emph{Sv\v{e}tlo a osv\v{e}tlen\'{i} - Osv\v{e}tlen\'{i} pracovn\'{i}ch prostor\r{u}: \v{C}\'{a}st 1: Vnit\v{r}n\'{i} pracovn\'{i} prostory.} Praha: \'{U}\v{r}ad pro technickou normalizaci, metrologii a st\'{a}tn\'{i} zku\v{s}ebnictv\'{i}, 2012

\bibitem{Habel} 
J. Habel. \textit{Sv\v{e}tlo a osv\v{e}tlov\'{a}n\'{i}}. Praha: FCC Public, 2013, 622~s. ISBN 978-80-86534-21-3.

\end{thebibliography}

\end{document}
