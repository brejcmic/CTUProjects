\section{Symmetric Solutions}
One of the requirements to the output design was to get the symmetric solutions of luminaire placement. There seemed to be two approaches how this might be done. The First counts with introduction of the symmetry in the fitness function. On the basis of the fitness equation evaluation the algorithm can prefer symmetric solutions more than others. Unfortunately this approach was very difficult due to unknown function that could describe how good is the symmetry of the luminare placement. Best experience came from using least squares method. After adding the sum of squares of the differences from the average value of ilumminance, a lot of output results showed the symmetry towards axis or center. However some types of luminous intensity distribution curves were very sensitive to create a tight groups of lamps in specific positions. This type of solutions were unable to realize in real conditions.
Authors chose eventually the second approach dealing with introduction of the symmetry in the dna.