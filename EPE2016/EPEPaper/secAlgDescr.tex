\section{Algorithm Description}
\subsection{Genetic Algorithm}
\label{ssec:GenAlg}
The algorithm must determine positions and number of luminaries in dependency on target illuminance and uniformity. This is the multicriteria and multiparametric type of the problem. The genetic algorithm offer quite simple way how to solve it, therefore they were used in this case. The genetic algorithms are well known today so only specific settings are further described.

The best solution was saved (elitism) from every population in two specimens. The first one was unable to change its DNA via mutation, the second one had the same probability of mutation like other solutions. Other parent solutions were selected via tournament selection. It consisted of making random group of $4$ solutions from the population and take the one with the best fitness. This type of selection had vital role for the algorithm. It avoided premature convergence of the best solution in comparison with the roulette selection. Similar effect was ensured also by recombination probability set less than $1$. There was used just one point crossover during the tests. Overview of the all GA settings is shown in table~\ref{tab:GAsettings}.

\begin{table}[htb]
	\renewcommand{\arraystretch}{1.3}
	\caption{Genetic Algorithm Settings}
 	\label{tab:GAsettings}
	\centering
  \begin{tabular}{| l | c |}
    \hline
    \textbf{First Population} & Random logic vectors \\
    \hline
    \textbf{Termination Cond.} & Maximum number of generations \\
    \hline
		\textbf{Number of Gen.} & $30$ \\
    \hline
		\textbf{Population Size} & $50$ \\
    \hline
		\textbf{Recombination Prop.} & $90 \%$ \\
    \hline
		\textbf{Mutation Prop.} & $4 \%$ \\
    \hline
		\textbf{Parent Selec.} & Tournament 1 of 4 \\
    \hline
		\textbf{Mutation Mech.} & Inverted bit \\
    \hline
		\textbf{Survival Selec.} & Elitism \\
    \hline
  \end{tabular}
\end{table}

\subsection{Fitness Function}
\label{ssec:FitFn}
The fitness function defines how good the solutions are. The target value of average illuminance and target value of uniformity are watched in the algorithm. In common case the average illuminance is given especially by number of the luminaires. The uniformity is given especially by placement of the luminaires on the other hand. The count of the luminaires is proportional to the investment cost to the lighting system. So the number of luminaires that exactly fulfill the target average value of illuminance is appropriate. The uniformity is always required as much as possible for defined number of luminaires. Therefore the fitness function was determined within discussed facts as follows:

\begin{equation}
\label{eq:fitness}
f_{DNA}\left(\overline{E}_{m}, U_0\right) = g_1\left(\overline{E}_{m}\right) + g_2\left(U_0\right)
\end{equation}

\begin{subnumcases}{\label{eq:fitnessG1} g_1\left(\overline{E}_{m}\right)=} 
  e^{\frac{\overline{E}_{m}-\overline{E}_{mT}}{\overline{E}_{m}}} &, $\left\langle 0, \overline{E}_{mT}\right\rangle$ \label{eq:fitnessG1A}\\
  e^{\frac{\overline{E}_{mT}-\overline{E}_{m}}{\overline{E}_{m}}} &, $\left( \overline{E}_{mT}, \infty\right)$ \label{eq:fitnessG1B}
\end{subnumcases}

\begin{subnumcases}{\label{eq:fitnessG2} g_2\left(U_0\right)=} 
  \frac{U_0}{2\cdot U_{0T}} &, $\left\langle 0, U_{0T}\right\rangle$ \label{eq:fitnessG2A}\\
  1-\frac{e^{U_{0T}-U_0}}{2} &, $\left( U_{0T}, \infty\right)$ \label{eq:fitnessG2B}
\end{subnumcases}

where:
\begin{description}
	\item[$\overline{E}_{m}$] is a calculated average value of illuminance,
	\item[$\overline{E}_{mT}$] is a target average value of illuminance,
	\item[$U_0$] is a calculated uniformity,
	\item[$U_{0T}$] is a target uniformity.
\end{description}

The function $g_1\left(\overline{E}_{m}\right)$ has peak value equal to $1$ at target value of average illuminance. Because the illuminance cannot be less than zero, the function reaches two limits:
\begin{equation}
\label{eq:g1lim0}
\lim_{\overline{E}_{m}\to 0+} g_1\left(\overline{E}_{m}\right) = 0
\end{equation}
\begin{equation}
\label{eq:g1limInf}
\lim_{\overline{E}_{m}\to \infty} g_1\left(\overline{E}_{m}\right) = e^{-1}
\end{equation}

Both limits have different values and the limit in infinity is higher than that in the $0$. This means that it is preferred the solution with the higher average illuminance for the same absolute difference from the target value.

\begin{figure}[htb]
  \centering
  \includegraphics[width=\columnwidth]{obrG1G2}
  \caption{Graphs of parts $g_1\left(\overline{E}_{m}\right)$ and $g_2\left(U_0\right)$ from the fitness function}
  \label{fig:fitG1G2}
\end{figure}

The function $g_2\left(U_{0}\right)$ reach the value $0.5$ at target value of uniformity. It has a bound at $0$ for values less then target value. There is a limit for values in interval higher than target value:
\begin{equation}
\label{eq:g2limInf}
\lim_{U_{0}\to \infty} g_2\left(U_{0}\right) = 1
\end{equation}

So the function $g_2\left(U_{0}\right)$ has an horizontal asymptote equal to one. Function $g_2\left(U_{0}\right)$ is linear for values of uniformity less than the target value. The highest slope is obtained here. For higher values of uniformity is the slope smaller due to the saturation effect of the exponential function. Therefore the algorithm is forced to reach target value of uniformity because of big change in the fitness function. Higher values makes the fitness better too, but there is a smaller effect.
Both functions $g_1\left(\overline{E}_{m}\right)$ and $g_2\left(U_{0}\right)$ are shown in figure~\ref{fig:fitG1G2} for better understanding.