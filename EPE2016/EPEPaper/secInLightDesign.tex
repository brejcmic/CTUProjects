\section{Interior Lighting Design Considerations}
%symetrie rozmístění svítidel, požadavky 12464, zvolená místnost 5*10 metrů, 500 lx bez oken, odraznosti započítány
Designing interior lighting systems for indoor working places requires fulfilling two contradictory criteria, namely providing enough light for persons occupying the given room at a reasonable power consumption. These and more parameters have been taken into account while composing standards such as \cite{CSN EN 12464-1}, being mandatory on the territory of the Czech Republic.

For this project an administrative model room has been chosen of dimensions $5 \times 10 $ meters, $4$ meters high with luminaires 3.5 metres above the floor. The model room's purpose has been chosen to be handwriting, writing on typewriters, reading and processing data according to reference number 5.26.2 of \cite{CSN EN 12464-1}. For this kind of room there are several conditions that have to be met by the lighting system:

\begin{description}
	\item[$\overline{E}_{m}$] Maintained Average Illuminance of 500 lux
	\item[$UGR_{L}$] Unified Glare Rating 19
	\item[$U_{0}$] Lighting Uniformity 0.6
	\item[$R_{a}$] General Color Rendering Index 80
\end{description}

To meet the requirements set by \cite{CSN EN 12464-1} for the model room, reference plane's average illuminance must be $\overline{E}_{m}$ or greater at all times during time of operation. The reference plane is defined as a horizontal plane 75 cm above the floor. $UGR_{L}$ will not be calculated in this project although this parameter could be critical for larger rooms. $R_{a}$ is a parameter of light sources and luminaires and thus must not be incorporated into calculations.