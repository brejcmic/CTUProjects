\section{Interior Lighting Design Considerations} \label{sec:design}
%symetrie rozmístění svítidel, požadavky 12464, zvolená místnost 5*10 metrů, 500 lx bez oken, odraznosti započítány
Designing interior lighting systems for indoor working places requires fulfilling two contradictory criteria, i.e. providing enough light for persons occupying the given room at a reasonable power consumption. These and more parameters have been taken into account while composing standards such as \cite{12464}, being mandatory on the territory of the Czech Republic.

For this project an administrative model room has been chosen of dimensions $5 \times 10 $ meters, $4$ meters high with luminaires 3.5 metres above the floor. The model room's purpose has been chosen to be handwriting, writing on typewriters, reading and processing data according to reference number 5.26.2 of \cite{12464}. For this kind of room there are several conditions that have to be met by the lighting system:

\begin{description}
	\item[$\overline{E}_{m}$] Maintained Average Illuminance of 500 lux
	\item[$UGR_{L}$] Unified Glare Rating 19
	\item[$U_{0}$] Lighting Uniformity 0.6
	\item[$R_{a}$] General Color Rendering Index 80
\end{description}

To meet the requirements set by \cite{12464} for the model room, reference plane's average illuminance must be $\overline{E}_{m}$ or greater at all times during over the course of operation. To calculate the initially needed illuminance values, the Maintenance Factor has to be calculated \cite{CIE97}. For this instance, MF has been chosen to be $0.75$. The reference plane is defined as a horizontal plane 75 cm above the floor for generic office tables as suggested in \cite{12464}. $UGR_{L}$ will not be included in calculations of this project, for the task area and view directions of users are unknown. $R_{a}$ is a parameter of light sources and luminaires and thus must not be incorporated into calculations.