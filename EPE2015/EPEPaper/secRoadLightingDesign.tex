\section{Road Lighting Design}
\label{sec:Road_Lighting_Design}
Lighting system designs of public roads in the Czech Republic have to meet among others the Czech Technical Standard \cite{CSN_EN_13201-2}. For this project a road for mainly cyclists but also for pedestrians has been chosen, that is according to \cite{CSN_EN_13201-1} lighting situation C1 and lighting class S4. Road lighting conforming to this class must meet in terms of \cite{CSN_EN_13201-2} the requirement of maintained average horizontal illuminance $\overline{E}_{M}\geq 5$~\text{lx}, minimum maintained illuminance $E_{min,M}\geq 1$~\text{lx} and uniformity so that the actual $\overline{E}_{M}$ must not exceed 1.5 times the minimal allowed $\overline{E}_{M}$ for the given lighting class, which is for class S4 $\overline{E}_{max,M} = 7.5$~\text{lx}.

The maintenance factor $MF$ is another entry that has to be taken into account. This factor defines the depreciation of the design level, in this case the depreciation of illuminance $\overline{E}$ and $E_{min}$ over the course of operation of the road lighting. Luminaire Atos from Schr\'{e}der has been chosen to be used in this project, its maintenance factor can be calculated from \cite{CSN_EN_13201-2_Z1}:
\begin{equation} \label{eq:MF}
MF = LLMF \cdot LSF \cdot LMF
\end{equation}
Where:
\begin{description}
	\item[$LLMF$] is the lamp lumen maintenance factor
	\item[$LSF$] is the lamp survival factor
	\item[$LMF$] is the luminaire factor
\end{description}

The Atos luminaire's are fitted with high pressure sodium lamps. For this project Osram NAV-T 70 W super 6Y has been chosen to operate for a period of 12000 hours before replacement. Looking up the product datasheet \cite{Osram} of the 70 W Osram lamp, $LLMF=0.89$ and $LSF=0.97$ can be obtained. The IP Code of the luminaire's optical system is IP66, meaning that for an environment of medium air pollution and cleaning intervals of 2 years $LMF=0.89$ can be found in table NA.1 of \cite{CSN_EN_13201-2_Z1}. According to equation~\ref{eq:MF} the final maintenance factor can be calculated as follows:

\begin{align}
&MF = 0.89 \cdot 0.97 \cdot 0.89 = 0.785603
\end{align}

To obtain design illuminances of $\overline{E}_{M}$ and $E_{min,M}$ at the beginning of the road lighting operation, maintenance factors have to be taken into account:
\begin{equation}
E = E_{M} \div MF
\end{equation}
Where:
\begin{description}
	\item[$E$] is the general illuminance value calculated or measured at the beginning of the operation cycle
	\item[$E_{M}$] is the maintained illuminance value assumed at the end of the operation cycle
	\item[$MF$] is the maintenance factor
\end{description}

Having calculated the maintenance factor, design illuminances can be specified:

\begin{align}
&\overline{E} = \overline{E}_{M} \div MF \doteq 6.36 \text{ lx}\\
&\overline{E}_{max} = \overline{E}_{max,M} \div MF \doteq 9.55 \text{ lx}\\
&\overline{E}_{min} = \overline{E}_{min,M} \div MF \doteq 1.27 \text{ lx}\\
\end{align}
